% Created 2017-02-13 Mon 15:56
% Intended LaTeX compiler: pdflatex
\documentclass[11pt]{article}
\usepackage[utf8]{inputenc}
\usepackage[T1]{fontenc}
\usepackage{graphicx}
\usepackage{grffile}
\usepackage{longtable}
\usepackage{wrapfig}
\usepackage{rotating}
\usepackage[normalem]{ulem}
\usepackage{amsmath}
\usepackage{textcomp}
\usepackage{amssymb}
\usepackage{capt-of}
\usepackage{hyperref}
\author{Tristan Strange}
\date{\today}
\title{}
\hypersetup{
 pdfauthor={Tristan Strange},
 pdftitle={},
 pdfkeywords={},
 pdfsubject={},
 pdfcreator={Emacs 25.1.1 (Org mode 9.0.3)}, 
 pdflang={English}}
\begin{document}

\tableofcontents

\section{Caricature}
\label{sec:org10e9e78}
\subsection{Algorithm}
\label{sec:orgb7b1c35}

\begin{verbatim}
# FeaturesOfMedium is set of functions that report features of input
FeaturesOfMedium[] <- getFeaturesOfMedium

# Set of inspiring examples - provides sense of normality
InspiringSet[] <- getInspiringSet

# the example we are actually going to caricature
SourceExample <- getSourceExample

# Get all features for inspiring set
for f in FeaturesOfMedium:
  for i in InspiringSet:
      Features[f][i] <- f(i)

  # average them and check distance from our source example's features
  FeatureDiffs[f] <- f(SourceExample) - Mean(Features[f])

# Get list of n most unusual features 
MostUnusualFeatures <- take(N, sort(FeatureDiffs))

# Construct fitness function that uses n most important features
# - Either disregard less different ones
# - or weight most unusual features more important

# Use the fitness function with some mere generation/search
Artifacts[] <- []
do some search that increases Fitness(Artifact)
    Artifact = MereGeneration(SourceExample, Artifacts)
    add Artifact to Artifacts
\end{verbatim}
\end{document}
